% ---------------------------------------------------------------------------- %
% PRELIMINARIES
% ---------------------------------------------------------------------------- %
\section{Preliminaries} \label{sec:preliminaries}

\subsection{The I/O Model}
Model and sorting bound... \cite{Aggarwal87}

\begin{table}[ht!]
  \centering
  
  \begin{tabular}{c|c|c}
           & time         & I/O
    \\ \hline
    Linear & $N$          & $N/B$
    \\
    Sort   & $N \log_2 N$ & $\tfrac{N}{B} \cdot \log_{M/B} \tfrac{N}{B}$
  \end{tabular}
  
  \caption{Comparison of common time bounds and their respective I/O bounds}
  \label{tab:time_vs_io}
\end{table}

Priority Queues... \cite{Arge04}

\subsubsection{Cache-oblivious Algorithms}
Tall cache assumption ...

Priority Queues... \cite{Arge07, Sanders2001}

\subsection{Ordered Boolean Decision Diagrams}
Definition and recursive solution with memoization table \cite{Bryant86,
  Brace90, Dijk16}.

Reduced OBBD's \cite[Definition2]{Bryant86}

Complementary edges \cite{Brace90}

Use of memoization table and motivation for I/O efficiency \todocite


\subsubsection{I/O Bounds of OBDDs}
\begin{theorem}[\cite{Arge96}] \label{thm:reduce_io_lower_bound}

  Reduction of an OBBD $G$ with minimal pair, level, depth first or breadth
  first blocking requires $\Omega(\sort(N))$ I/Os in the worst case.
\end{theorem}

\begin{theorem}[\cite{Arge96}] \label{thm:apply_io_worst_case}

  The Dynamic Programming \Apply\ algorithm on two OBDDs of size $N_1, N_2$
  followed up by a \Reduce\ operation requires $O(N_1 \cdot N_2)$ I/Os in the
  worst case.
\end{theorem}




%%% Local Variables:
%%% mode: latex
%%% TeX-master: "main"
%%% End:
